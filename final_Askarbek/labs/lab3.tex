\chapter{Lab3}

\begin{multicols}{2}

Section 2.2 Continuity 67

  Applied to Example 2.2.16, Corollary 2.2.13 implies that the function $g(x) =  cos1 \backslash x$  is uniformly continuous on $[\rho ,1]$ if  $0<\rho < 1.$



        More About Monotonic Functions

\begin{theorem} \label{t220}
 implies that if f is monotonic on an interval I , then f is either continuous
 or has a jump discontinuity at each x0 in I . This and \ref{t2214} provide the key to
 the proof of the following theorem.
\end{theorem}

\begin{theorem}\label{t221}
 If f is monotonic and non constant on $[a b]$ , then f is continuous on
$[a, b]$  if and only if its range  $\tilde{R} _f = \big\{f(x) \parallel x  \in(a,b)  \big\} $ is the closed interval with endpoints $f(a)$ and $f(b).$
\end{theorem}




\begin{proof}
We assume that f is non decreasing, and leave the case where f is non increasing to you (Exercise 34). \ref{t220} implies that the set$ \tilde{R}_f $
is a
subset of the open interval $(f(a+),f(b-)).$ Therefore,
\end{proof}


\begin{equation}\label{a15}
\begin{gathered}
R_f = \big\{f(a)\big\} \cup \tilde{R}_f \cup  \big\{f(b)\big\}  \subset  \big\{f(a) \big\}  \cup  \big\{f(a+)           \\ \big\},\big\{f(b-)\big\} \cup \big\{f(b)\big\}
\end{gathered}
\end{equation}

Now suppose that f is continuous on $[a,b]$. Then $f (a) = f(a+),f(b-) = f(b)$, so (\ref{a15})
implies that$ R_f \subset [f(a),f(b)].If f(a)< \mu <f(b)$ , then \ref{t221} implies that $\mu = f(x) for some x in (a, b). Hence, R_f =[f(a),f(b)].$


For the converse, suppose that $R_f = [f(a),f(b)].$ Since $f(a) \leq f(a+) and f(b-) \leq f(b),$ \ref{a15} implies that $f(a) = f(a+) and f(b-).$ We know from \ref{t219} that if f is non-decreasing and $a < x_0 < b$ ,then


$$f(x_0 -) \leq f(x_0) \leq f(x_0 -).$$


If either of these inequalities is strict, $Rf$ cannot be an interval. Since this contradicts our
assumption.$f(x_0 -) = f(x_0) = f(x_0 +_)$.Therefore,f is continuous at $x_0$(Exercise(2).
We can now conclude that f is continuous on $[a,b]$.




\ref{t221} implies the following theorem



\begin{theorem}\label{t2215}
 Suppose that f is increasing and continuous on $[a, b]$ and let $f(a)= c$ and $f (b) = d $ Then there is a unique function g defined on $[c,d]$ such that
\end{theorem}






\begin{equation}\label{a16}
g(f(x)) = x, a \leq x \leq b,
\end{equation}

and 

\begin{equation}\label{a17}
f(g(y) = y,\ c \leq y \leq d.
\end{equation}


Moreover, g is continuous and increasing on $[c,d]$.



\begin{proof}
We first show that there is a function g satisfying (\ref{a16}) and (\ref{a17}). Since f is continuous, \ref{t2215} implies that for each $y_0$ in $[c,d]$ there is an $x_0$ in $[a, b]$ such that
\end{proof}



\begin{equation}\label{a18}
f(x_0) = y(0),
\end{equation}  

\end{multicols}
