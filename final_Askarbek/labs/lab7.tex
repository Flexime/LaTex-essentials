
\chapter{Lab7}

%\title{A Data Mining Approach for Visual and Analytical
%Identification of Neurorehabilitation Ranges in Traumatic
%Brain Injury Cognitive Rehabilitation}
%\author{Alejandro García-Rudolph^\ref{l1} \ref{l2} \ref{l3}  and Karina Gibert^\ref{l4}}
%\date{2014}



\begin{flushleft}
\begin{enumerate}
\item Institut Guttmann, Institut Universitari de Neurorehabilitacio adscrit a la UAB, Badalona, Barcelona, Spain \label{l1}
\item Universitat Autonoma de Barcelona, Bellaterra, Cerdanyola del Vall ` es, Spain \label{l2}
\item Fundacio Institut dInvestigaci  o en Ci  encies de la Salut Germans Trias i Pujol, Badalona, Barcelona, Spain\label{l3}
\item Departament d’Estad´ıstica i Investigacio Operativa, Universitat Polit ´ ecnica de Catalunya (BarcelonaTech), Jordi Girona
08034 Barcelona, Spain\label{l4}
\end{enumerate}

Correspondence should be addressed to Alejandro Garc´ıa-Rudolph; alejandropablogarcia@gmail.com \newline

Received 2 January 2015; Accepted 23 February 2015 \newline

Academic Editor: Francisco Solis \newline



Copyright \copyright 2015 A. Garc´ıa-Rudolph and K. Gibert. This is an open access article distributed under the Creative Commons
Attribution License, which permits unrestricted use, distribution, and reproduction in any medium, provided the original work is
properly cited.\newline 
Traumatic brain injury (TBI) is a critical public health and socioeconomic problem throughout the world. Cognitive rehabilitation
(CR) has become the treatment of choice for cognitive impairments after TBI. It consists of hierarchically organized tasks that
require repetitive use of impaired cognitive functions. One important focus for CR professionals is the number of repetitions and
the type of task performed throughout treatment leading to functional recovery. However, very little research is available that
quantifies the amount and type of practice. The Neurorehabilitation Range (NRR) and the Sectorized and Annotated Plane (SAP)
have been introduced as a means of identifying formal operational models in order to provide therapists with decision support
information for assigning the most appropriate CR plan. In this paper we present a novel methodology based on combining SAP
and NRR to solve what we call the Neurorehabilitation Range Maximal Regions (NRRMR) problem and to generate analytical
and visual tools enabling the automatic identification of NRR. A new SAP representation is introduced and applied to overcome
the drawbacks identified with existing methods. The results obtained show patterns of response to treatment that might lead to
reconsideration of some of the current clinical hypotheses.
\end{flushleft}





\newpage

\twocolumn{

\section{Introduction} 

\normalsize


Traumatic brain injury (TBI) is a critical public health and
socioeconomic problem throughout the world. Although
high-quality prevalence data are scarce, it is estimated that
in the USA around 5.3 million people are living with a TBIrelated disability, and in the European Union approximately
7.7 million people who have experienced a TBI have disabilities \cite{biblref1}.
TBI is considered a silent epidemic, because society is
largely unaware of the magnitude of the problem \cite{biblref2}. The
World Health Organization predicts that, by the year 2020,
TBI and road traffic accidents will be the third greatest cause
of disease and injury worldwide \cite{biblref3}.


The consequences of TBI vary from case to case but
can include motor, cognitive, and behavioral deficits in the
patient, disrupting their daily life activities at personal, social,
and professional levels. The most important cognitive deficits
after suffering a TBI are those related to attention, decrease
in memory and learning capacity, worsening of the capacity
to schedule and to solve problems, a reduction in abstract
thinking, communication problems, and a lack of awareness
of one’s own limitations. These cognitive impairments hamper the path to functional independence and a productive
lifestyle for the person with TBI.
New techniques of early intervention and the development of intensive TBI care have improved the survival rate
noticeably. However, despite these advances, brain injuries


still have no surgical or pharmacological treatment to reestablish lost functions \cite{biblref4}. In this context, cognitive rehabilitation
(CR) is defined as a process whereby people with brain
injury work together with health service professionals and
others to remedy or alleviate cognitive deficits arising from
a neurological injury \cite{biblref5}.


The structure of the paper is as follows: \ref{sec1} briefly
presents the state of the art and the starting point of
the proposal. \ref{sec2} introduces the proposed analysis
methodology and \ref{sec3} its application to the CR context;
\ref{sec4} presents a discussion of the obtained results and a
comparison with the previous and \ref{sec5} the conclusions
and future lines of research.


The process by which neuronal circuits are modified by
experience, learning, or injury is referred to as neuroplasticity
\cite{biblref7}.
While task repetition is not the only important feature,
it is becoming clear that neuroplastic change and functional
improvement occur after a number of specific tasks are
performed but do not occur with other numbers \cite{biblref8,biblref9} Thus,
one important focus for rehabilitation professionals is the
number of repetitions and the type of task performed during
treatment. However, there is very little research to quantify
the amount and type of practice that occurs during clinical
rehabilitation treatment and its relationship to rehabilitation
outcomes \cite{biblref10,biblref11}.



\section{State of the Art} 
\label{sec1}


There is a common belief that CR is effective for TBI patients,
based on a large number of studies and extensive clinical
experience. Different statistical methodologies and predictive
data mining methods have been applied to predict clinical
outcomes of the rehabilitation of patients with TBI \cite{biblref14,biblref16} Most of these studies focus on determining survival,
predicting disability or the recovery of patients, and looking
for the factors that better predict the patient’s condition after
TBI

\section{Materials and Methods}
\label{sec2}
The proposed methods present two strategies for the analytical and graphical identification and visualization of NRR and
non-NRR based on the notion of SAP as introduced in \cite{biblref12}
and on the classical MER problem, respectively.

\textit{3.1.Sectorized} and Annotated Plane (SAP). Given three
variables $y$, $x_1$, and $x_2$, where $y$ is a qualitative response
variable, with values ${\{y_1,y_2,...}\}$, and $x_1, y_2$ numerical
explanatory variables, the SAP is a 2-dimensional plot with
$x_1$ in the axis, $x2$ in the -axis and rectangular regions
with constant $y$ displayed and labeled with $y$ values as
outlined in Figure 1. An SAP is therefore a graphical support
tool aimed at visualization, where the response variable is
constant in certain regions of the $ x_1  \backslash \text{times}  y_2 $ space. Eventually,
allowing a relaxation of strict constant in the marked
regions, the SAP might include an indicator of region purity,
adding the probability of occurrence of the labeling value.

\section{Application and Results}
\label{sec3}
\textit{4.1 Clinical Context. This work is based on the same context
as in \cite{biblref12}, the Neuropsychological Department of Institut
Guttmann Neurorehabilitation Hospital (IG). The Information Technology framework for CR treatments in this clinical
setting is therefore the PREVIRNEC \copyright platform \cite{biblref31}. It is
specifically designed to operate CR plans assigned to subjects,
as well as to manage precise follow-up information about the
process.}

\section{Discussion}
\label{sec4}

This work aims to identify the conditions in which performing a certain cognitive rehabilitation task (or a group of
tasks) guarantees better potential for the activation of brain
plasticity and therefore helps bring about improvements in
the assessed cognitive functions after CR treatment. As this
research takes our previous research as a starting point, the
results comparison is provided below and the pros and cons
are discussed

\section*{Conclusions and Future Work}
\label{sec5}

This work builds on our previous contribution towards the
design, implementation, and execution of personalized, predictable, and data-driven CR programs. We wish to identify
NRR for cognitive rehabilitation tasks that lead to patient
improvement.
\section*{Conflict of Interests}
\label{sec6}
No competing financial interests exist
\section{Acknowledgments}
\label{sec7}

This research was supported by the Ministry of Science and
Innovation (Spain) INNPACTO Program (PT NEUROCONTENT, Grant no. 300000-2010-30), Ministry of Education
Social Policy and Social Services (Spain) IMSERSO Program
(PT COGNIDAC, Grant no. 41/2008), MARATO TV3 Foun- ´
dation (PT: Improving Social Cognition and Meta-Cognition
in Schizophrenia: A Tele-Rehabilitation Project, Grant no.
091330), EU CIP-ICT-PSP-2007-1 (PT: CLEAR, Grant no.
224985), Spanish Ministry of Economy and Finance (PT
COGNITIO, Grant no. TIN2012 38450), and EU-FP7-ICT
(PT PERSSILAA Grant no. 610359). }
