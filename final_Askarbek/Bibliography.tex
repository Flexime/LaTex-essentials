
\addcontentsline{toc}{chapter}{References}


\begin{thebibliography}{31}
\bibitem{biblref1} 
B. Roozenbeek, A. I. R. Maas, and D. K. Menon, “Changing
patterns in the epidemiology of traumatic brain injury,” Nature
Reviews Neurology, vol. 9, no. 4, pp. 231–236, 2013.

\bibitem{biblref2} 
 J. A. Langlois and R. W. Sattin, “Traumatic brain injury in the
United States: research and programs of the Centers for Disease
Control and Prevention (CDC),” Journal of Head Trauma
Rehabilitation, vol. 20, no. 3, pp. 187–188, 2005.

\bibitem{biblref3} 
S. A. Tabish and N. Syed, “Traumatic brain injury: the neglected
epidemic of modern society,” International Journal of Science
and Research, vol. 3, no. 12, 2014

\bibitem{biblref4} 
D. T. Stuss, G. Winocur, and I. H. Robertson, Cognitive NeuroRehabilitation: Evidence and Application, Cambridge University
Press, Cambridge, UK, 2nd edition, 2008.

\bibitem{biblref5} 
B. A. Wilson, “La readaption cognitive chez les c ´ er´ ebro-l ´ es´ es,” in ´
Neuropsychologie Clinique et Neurologie du Comportement, M. I.
Botez, Ed., pp. 637–652, Les Presses de l’Universite de Montreal, ´
Montreal, Canada, 2nd edition, 1996.

\bibitem{biblref6} 
M. M. Sohlberg, Cognitive Rehabilitation. An interactive Neuropsychological Approach, 2001, edited by: C. A. Mateer.

\bibitem{biblref7} 
R. J. Nudo, “Adaptive plasticity in motor cortex: implications
for rehabilitation after brain injury,” Journal of Rehabilitation
Medicine, no. 41, supplement, pp. 7–10, 2003.

\bibitem{biblref8} 
J. R. Carey, W. K. Durfee, E. Bhatt et al., “Comparison of finger
tracking versus simple movement training via telerehabilitation
to alter hand function and cortical reorganization after stroke,”
Neurorehabilitation and Neural Repair, vol. 21, no. 3, pp. 216–
232, 2007.
\bibitem{biblref9} 
S. L. Wolf, C. J. Winstein, J. P. Miller et al., “Effect of constraint induced movement therapy on upper extremity function 3 to
9 months after stroke: the EXCITE randomized clinical trial,”
Journal of the American Medical Association, vol. 296, no. 17, pp.
2095–2104, 2006.

\bibitem{biblref10} 
C. K. English, S. L. Hillier, K. R. Stiller, and A. WardenFlood, “Circuit class therapy versus individual physiotherapy
sessions during inpatient stroke rehabilitation: a controlled
trial,” Archives of Physical Medicine and Rehabilitation, vol. 88,
no. 8, pp. 955–963, 2007.

\bibitem{biblref11} 
 S. Kuys, S. Brauer, and L. Ada, “Routine physiotherapy does not
induce a cardio respiratory training effect post-stroke, regardless
of walking ability,” Physiotherapy Research International, vol. 11,
no. 4, pp. 219–227, 2006.

\bibitem{biblref12}
 A. Garc´ıa-Rudolph and K. Gibert, “A data mining approach
to identify cognitive NeuroRehabilitation Range in Traumatic
Brain Injury patients,” Expert Systems with Applications, vol. 41,
no. 11, pp. 5238–5251, 2014.

\bibitem{biblref13}  
A. Naamad, D. T. Lee, and W.-L. Hsu, “On the maximum empty
rectangle problem,” Discrete Applied Mathematics, vol. 8, no. 3,
pp. 267–277, 1984.

\bibitem{biblref14}  
A. I. Rughani, T. S. M. Dumont, Z. Lu et al., “Use of an artificial
neural network to predict head injury outcome,” Journal of
Neurosurgery, vol. 113, no. 3, pp. 585–590, 2010
\bibitem{biblref15}  
S.-Y. Ji, R. Smith, T. Huynh, and K. Najarian, “A comparative analysis of multi-level computer-assisted decision making
systems for traumatic injuries,” BMC Medical Informatics and
Decision Making, vol. 9, no. 1, article 2, 2009.
\bibitem{biblref16}  
 B. C. Pang, V. Kuralmani, R. Joshi et al., “Hybrid outcome
prediction model for severe traumatic brain injury,” Journal of
Neurotrauma, vol. 24, no. 1, pp. 136–146, 2007
\bibitem{biblref17}  
 M. L. Rohling, M. E. Faust, B. Beverly, and G. Demakis,
“Effectiveness of cognitive rehabilitation following acquired
brain injury: a meta-analytic re-examination of cicerone et al.’s
(2000, 2005) systematic reviews,” Neuropsychology, vol. 23, no.
1, pp. 20–39, 2009



\bibitem{bibleref18}
 J. Whyte and T. Hart, “It’s more than a black box; it’s a Russian
doll: defining rehabilitation treatments,” American Journal of
Physical Medicine  Rehabilitation, vol. 82, no. 8, pp. 639–652,
2003.
\bibitem{bibleref19}
K. D. Cicerone, D. M. Langenbahn, C. Braden et al., “Evidencebased cognitive rehabilitation: updated review of the literature
from 2003 through 2008,” Archives of Physical Medicine and
Rehabilitation, vol. 92, no. 4, pp. 519–530, 2011.
\bibitem{bibleref20}
 K. Gibert and A. Garc´ıa-Rudolph, “Desarrollo de herramientas
para evaluar el resultado de las tecnolog´ıas aplicadas al proceso rehabilitador Estudio a partir de dos modelos concretos:
Lesion Medular y Da ´ no Cerebral Adquirido,” in ˜ Posibilidades
de Aplicacion de Miner ´ ´ıa de Datos para el Descubrimiento
de Conocimiento a Partir de la Practica Cl ´ ´ınica, Informes de
Evaluacion de Tecnologias Sanitarias, AATRM n ´ um. 2006/11,
Cap 6, Plan Nacional para el Sistema Nacional de Salud del
Ministerio de Sanidad y Consumo, Madrid, Spain; Agencia `
d’Avaluacio de Tecnologia I Recerca M ´ ediques, Barcelona, `
Spain, 2007
\bibitem{bibleref21}
J. Serra, J. L. Arcos, A. Garcia-Rudolph, A. Garc´ıa-Molina, T.
Roig, and J. M. Tormos, “Cognitive prognosis of acquired brain
injury patients using machine learning techniques,” in Proceedings of the International Conference on Advanced Cognitive
Technologies and Applications (COGNITIVE ’13), pp. 108–113,
IARIA, Valencia, Spain, 2013
\bibitem{bibleref22}
 A. Marcano-Cedeno, P. Chausa, A. Garc ˜ ´ıa, C. Caceres, J. M. ´
Tormos, and E. J. Gomez, “Data mining applied to the cognitive ´
rehabilitation of patients with acquired brain injury,” Expert
Systems with Applications, vol. 40, no. 4, pp. 1054–1060, 2013.
\bibitem{bibleref23}
 V. Jagaroo, Neuroinformatics for Neuropsychologists, Springer,
1st edition, 2009.
\bibitem{bibleref24}
A. Dumitrescu and M. Jiang, “On the largest empty axis-parallel
box amidst points,” Algorithmica, vol. 66, no. 2, pp. 225–248,
2013
\bibitem{bibleref25}
B. Chazelle, R. L. Drysdale, and D. T. Lee, “Computing the
largest empty rectangle,” SIAM Journal on Computing, vol. 15,
no. 1, pp. 300–315, 1986.
\bibitem{bibleref26}
 A. Aggarwal and S. Suri, “Fast algorithms for computing the
largest empty rectangle,” in Proceedings of the 3rd Annual Symposium on Computational Geometry, pp. 278–290, Waterloo,
Canada, June 1987

\bibitem{bibleref27}
 M. McKenna, J. O’Rourke, and S. Suri, “Finding the largest
rectangle in an orthogonal polygon,” in Proceedings of the 23rd
Annual Allerton Conference on Communication, Control and
Computing, pp. 486–495, Urbana Champaign, Ill, USA, October
1985.
\bibitem{bibleref28}
H. S. Baird, S. E. Jones, and S. J. Fortune, “Image segmentation
by shape-directed covers,” in Proceedings of the 10th International Conference on Pattern Recognition, vol. 1, pp. 820–825,
June 1990.

\bibitem{bibleref29}
 J. Edmonds, J. Gryz, D. Liang, and R. J. Miller, “Mining for
empty spaces in large data sets,” Theoretical Computer Science,
vol. 296, no. 3, pp. 435–452, 2003.

\bibitem{bibleref30}
J. Augustine, S. Das, A. Maheshwari, S. C. Nandy, S. Roy,
and S. Sarvattomananda, “Recognizing the largest empty circle and axis-parallel rectangle in a desired location,” CoRR,
abs/1004.0558v2, 2010.


\bibitem{biblref31}
J. M. Tormos, A. Garcia-Molina, A. Garcia Rudolph, and T.
Roig, “Information and communications technology in learning development and rehabilitation,” International Journal of
Integrated Care, vol. 9, 2009.
\end{thebibliography}
