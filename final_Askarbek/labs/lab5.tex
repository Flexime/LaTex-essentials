\chapter{Lab5}
\begin{flushleft}


146 Chapter 3 Integral Calculus of Functions of One Variable

is an antiderivative of $f(\phi(t) \phi' (t) $ on  $[c,d]$
and \ref{t219} implies that

\begin{equation}
  \int_c^d   f(\phi(t))\phi't (t) dt = G(d) = F(\phi (d)) - F(\phi(c) )  
\end{equation}

        

\begin{center}
$=F( \beta ) - F( \alpha ).$
\end{center}
Comparing this with (22) yields (21).


\textbf{Example 3.3.2} To evaluate the integral


$$
I =  \int_{-1/\sqrt2}^{1/\sqrt2} (1 -2  x^{2}) (1 -2  x^{2})^{-1/2} dx
$$

we let 

$$
f(x) = (1 -2  x^{2}) (1 -2  x^{2})^{-1/2} , -1/\sqrt 2 \leq x \leq 1/\sqrt 2 , 
$$

and 

$$
x = \phi(t) = sin(t) , -\pi/4 \leq t \leq \pi/4. 
$$


Then $phi (t) = cos \backslash t $ and 

% says missing $ and {  but it correctly shows in pdf 
% I made it look like system of euations 
\begin{equation*}
\begin{cases}
    I =  \int_ {-1\sqrt2}^{1\sqrt2}  f(x) dx  =  \int_{-\pi/4}^{\pi/4} f(sint) cos\ t \ dt \\
= \int_{-\pi/4}^{\pi/4} (1-2sin^2 t)(1-2sin^2 t)^{-1/2} cos \ t  \ dt.  \\
(1-2sin^2 t)^{1/2} =    \\

cost,-\pi/4 \leq t \leq \pi/4 

\end{cases}
\end{equation*}







and 

$$
1-2sin^2 t = cos2t,
$$

(23) yields 


$$
I  =  \int_{-\pi/4}^{\pi/4} x  cos2\ tdt = \frac{sin2t}{2} \bigg|_{-\pi/4}^{\pi/4} =1.
$$



\textbf{Example 3.3.3} To evaluate the integral

$$
I  =  \int_{0}^{5\pi}   \frac{sin\ t }{2 + cos \ t } 
$$
 
 we take $\phi (t) = cost.$ Then $\phi' (t) = -sin \ t $



$$
I  =  -\int_{0}^{5\pi}   \frac{\phi'(t) }{2 + \phi(t) } dt =  - \int_{0}^{5\pi}f(\phi(t)) \phi'(t)dt , 
$$


where 
$$
f(x) = \frac{1}{2+x}.
$$



\newpage
\begin{flushright}
Section 4.1 Sequences of Real Numbers 189
\end{flushright}

\begin{proof}
 The existence and uniqueness of $\bar{s}$ and  $\underline{s}$  follow from \ref{t4112} and \ref{d221} If $\bar{s}$  and $\underline{s}$ are both finite, then \ref{a16} and \ref{a18} imply that
\end{proof}

$$
\underline{s} - \epsilon  < \bar{s} + \epsilon
$$

for every $\epsilon > 0 $, which implies (21).If $\underline{s} = - \infty $ or $\bar{s} = \infty $, then(21) is obvious.If $\underline{s} = \infty $ or  $\bar{s} = \infty $, then (21) follows immediately from Definition 4.1.10.


\textbf{Example 4.1.13}


$$
\overline{\lim_{n \rightarrow \infty }}  r^n = 
\begin{cases}
\infty , & |r|>1,\\
1, & |r|=1,\\
0, & |r|<1;
\end{cases} 
$$

and

$$
\underline{\lim_{n \rightarrow \infty }}  r^n =  
\begin{cases}
\infty , & r>1,\\
1, & r=1,\\
0, & |r|<1, \\
-1 ,& r= -1, \\
-\infty,& r<-1.
\end{cases} 
$$
Also,


$$
\overline{\lim_{n \rightarrow \infty }}  n^2 =   
\underline{\lim_{n \rightarrow \infty }} n^2 = \infty,
$$

$$
\overline{\lim_{n \rightarrow \infty }} (-1)^n (1- \frac{1}{n}) = 1 , \underline{\lim_{n \rightarrow \infty }} (-1)^n (n- \frac{1}{n}) = -1,
$$
and
$$
\overline{\lim_{n \rightarrow \infty }}[1+(-1)^n] n^2 = \infty , 
\underline{\lim_{n \rightarrow \infty }} [1+(-1)^n] n^2 =0.
$$

\begin{theorem}\label{t4112}
If $\{s_n\}$ is a sequence of real numbers ,then 
\end{theorem}

\begin{equation*}\label{a22}
\lim_{n \rightarrow \infty}  s_n = s \eqno(22)
\end{equation*}

if and only if 

\begin{equation*}\label{a23}
\overline{\lim_{n \rightarrow \infty}}  s_n  =
\underline{\lim_{n \rightarrow \infty}}  s_n =  s. \eqno(23)
\end{equation*}

\begin{proof}
If $s = \pm \infty $, the equivalence of (\ref{a22}) and (\ref{a23}) follows immediately from their
definitions. If  $\lim_{n \rightarrow \infty }  S_n = s$ (finite), then Definition 4.1.1 implies that (16)–(19)
hold with $\overline{s} $ and $underline{s}$ replaced by $s$. Hence, (\ref{a23}) follows from the uniqueness of $\overline{s} $ and  $underline{s}$. For the
converse, suppose that s D s and let s denote their common value. Then \ref{a16} and \ref{a18} imply that 
\end{proof}



$$
 s -  \epsilon  < s_n < s +  \epsilon 
$$

for large n, and (22) follows from Definition 4.1.1 and the uniqueness of $lim_{n \rightarrow \infty} s_n$ \ref{t219}.

\end{flushleft}