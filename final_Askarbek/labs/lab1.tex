\chapter{Lab1}
\begin{center} 
	1. Если c − постоянное число и функция f(x) интегрируема на
$[a;b]$, то 
$$\int\limits_{a}^{b} c \cdot f(x)dx = c \cdot \int\limits_{a}^{b} f(x)dx \eqno(2.4)$$


 
то есть постоянный множитель c можно выносить за знак
определенного интеграла.
 

 
	2. Если функции $f_1(x)$ и $f_2(x)$ интегрируемы на  $[a;b]$,тогда
интегрируема на [a;b] их алгебраическая сумма и
$$\int\limits_{a}^{b} (f_1(x)\pm f_2(x))dx = \int\limits_{a}^{b} (f_1(x)\pm \int\limits_{a}^{b} f_2(x))dx \eqno(2.5)$$
\


то есть интеграл от алгебраической суммы равен алгебраической сумме
интегралов. Это свойство распространяется на сумму любого конечного
числа слагаемых.



	3. При перестановке пределов интегрирования знак интеграла
изменяется на противоположный:
 
$$
\int\limits_{a}^{b} f(x)dx = - \int\limits_{a}^{b}f(x)dx. \eqno(2.6)
$$
\end{center}
